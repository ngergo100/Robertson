\documentclass[12pt]{report}
\usepackage[utf8]{inputenc}

% no word splitting
\tolerance=1
\emergencystretch=\maxdimen
\hyphenpenalty=10000
\hbadness=10000

% every section on a new page
\usepackage{titlesec}
\newcommand{\sectionbreak}{\clearpage}

\usepackage{bm}

% includegraphics
\usepackage{graphicx}

%Matrices
\usepackage{amsmath}
\usepackage{commath}

%Subfigure
\usepackage{subcaption}
\newcommand{\matr}[1]{\bm{#1}}

%No indent in a new paragraph
\usepackage[parfill]{parskip}

\begin{document}
	
	\section*{Task \#1}
	
	Let's say $f(t)=\sum_{i=1}^{3}y_i(t)$ if $t\geq 0$! Then 
	\begin{equation}
		f(t)=y_1(t)+y_2(t)+y_3(t)\,,
	\end{equation}
	and
	\begin{equation}
	\dot{f}(t)=\dot{y_1}(t)+\dot{y_2}(t)+\dot{y_3}(t)\,.
	\end{equation}
	The time derivatives are know from the given system. Let's substitute them back!
	\begin{equation}
	\dot{f}(t)=-\alpha y_1+\beta y_2y_3+\alpha y_1-\beta y_2y_3\gamma -y_{2}^2+y_{2}^2
	\label{eq:fdot}	
	\end{equation}
	From equation \ref{eq:fdot} it is clear that $\dot{f}(t)\equiv 0$, consequently $f(t)$ is a constant function. We know from the inital conditions that
	\begin{equation}
		f(0)=y_1(0)+y_2(0)+y_3(0)=1+0+0=1
	\end{equation}
	Since $f$ is a constant function, $f(t)=f(0)=1$.
	
	We can find the steady state if we solve
	\begin{equation}
		\matr{\dot{y}}=\matr{0} \,,
	\end{equation}
	which means these three equations
	\begin{equation}
		\begin{aligned}
		-\alpha y_1 + \beta y_2 y_3 = 0 \\
		\alpha y_1 - \beta y_2 y_3 - \gamma y_2^2 = 0 \\
		\gamma y_2^2 = 0
		\end{aligned} \,.
	\end{equation}
	From the third equation we obtain $y_2=0$. If $y_2=0$ then $y_1=0$ according to the first equation. $y_3$ could be anything, however we know that $\sum_{i=1}^{3}y_i(t)=1$. Since $y_1=y_2=0$, $y_3$ must equal to 1. So the steady state is
	$$
	\matr{y_s}=\begin{pmatrix}
		0 \\ 
		0\\ 
		1
	\end{pmatrix}
	$$

	\section*{Task \#2}
	
	
	\section*{Task \#3}
	
\end{document}